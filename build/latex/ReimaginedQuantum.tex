%% Generated by Sphinx.
\def\sphinxdocclass{report}
\documentclass[letterpaper,10pt,english]{sphinxmanual}
\ifdefined\pdfpxdimen
   \let\sphinxpxdimen\pdfpxdimen\else\newdimen\sphinxpxdimen
\fi \sphinxpxdimen=49336sp\relax

\usepackage[margin=1in,marginparwidth=0.5in]{geometry}
\usepackage[utf8]{inputenc}
\ifdefined\DeclareUnicodeCharacter
  \DeclareUnicodeCharacter{00A0}{\nobreakspace}
\fi
\usepackage{cmap}
\usepackage[T1]{fontenc}
\usepackage{amsmath,amssymb,amstext}
\usepackage{babel}
\usepackage{times}
\usepackage[Bjarne]{fncychap}
\usepackage{longtable}
\usepackage{sphinx}

\usepackage{multirow}
\usepackage{eqparbox}

% Include hyperref last.
\usepackage{hyperref}
% Fix anchor placement for figures with captions.
\usepackage{hypcap}% it must be loaded after hyperref.
% Set up styles of URL: it should be placed after hyperref.
\urlstyle{same}
\addto\captionsenglish{\renewcommand{\contentsname}{Contents:}}

\addto\captionsenglish{\renewcommand{\figurename}{Fig.\@ }}
\addto\captionsenglish{\renewcommand{\tablename}{Table }}
\addto\captionsenglish{\renewcommand{\literalblockname}{Listing }}

\addto\extrasenglish{\def\pageautorefname{page}}

\setcounter{tocdepth}{1}



\title{Reimagined Quantum Documentation}
\date{Apr 12, 2017}
\release{1.0.0}
\author{Juan Barbosa}
\newcommand{\sphinxlogo}{}
\renewcommand{\releasename}{Release}
\makeindex

\begin{document}

\maketitle
\sphinxtableofcontents
\phantomsection\label{\detokenize{index::doc}}



\chapter{Documentation}
\label{\detokenize{code::doc}}\label{\detokenize{code:welcome-to-reimagined-quantum-s-documentation}}\label{\detokenize{code:documentation}}

\section{ReimaginedQuantum library}
\label{\detokenize{code:reimaginedquantum-library}}\label{\detokenize{code:module-reimaginedQuantum}}\index{reimaginedQuantum (module)}
Created on Mon Apr 10 11:19:25 2017

@author: juan
\index{Channel (class in reimaginedQuantum)}

\begin{fulllineitems}
\phantomsection\label{\detokenize{code:reimaginedQuantum.Channel}}\pysiglinewithargsret{\sphinxstrong{class }\sphinxcode{reimaginedQuantum.}\sphinxbfcode{Channel}}{\emph{name}, \emph{port}}{}
Constants
\index{END\_COMMUNICATION (reimaginedQuantum.Channel attribute)}

\begin{fulllineitems}
\phantomsection\label{\detokenize{code:reimaginedQuantum.Channel.END_COMMUNICATION}}\pysigline{\sphinxbfcode{END\_COMMUNICATION}\sphinxstrong{ = 4}}
End of message

\end{fulllineitems}

\index{READ\_VALUE (reimaginedQuantum.Channel attribute)}

\begin{fulllineitems}
\phantomsection\label{\detokenize{code:reimaginedQuantum.Channel.READ_VALUE}}\pysigline{\sphinxbfcode{READ\_VALUE}\sphinxstrong{ = 14}}
Reading operation signal

\end{fulllineitems}

\index{START\_COMMUNICATION (reimaginedQuantum.Channel attribute)}

\begin{fulllineitems}
\phantomsection\label{\detokenize{code:reimaginedQuantum.Channel.START_COMMUNICATION}}\pysigline{\sphinxbfcode{START\_COMMUNICATION}\sphinxstrong{ = 2}}
Begin message signal

\end{fulllineitems}

\index{WRITE\_VALUE (reimaginedQuantum.Channel attribute)}

\begin{fulllineitems}
\phantomsection\label{\detokenize{code:reimaginedQuantum.Channel.WRITE_VALUE}}\pysigline{\sphinxbfcode{WRITE\_VALUE}\sphinxstrong{ = 15}}
Writing operation signal

\end{fulllineitems}

\index{construct\_message() (reimaginedQuantum.Channel method)}

\begin{fulllineitems}
\phantomsection\label{\detokenize{code:reimaginedQuantum.Channel.construct_message}}\pysiglinewithargsret{\sphinxbfcode{construct\_message}}{\emph{read=False}}{}
Construcs a message with project requirements.
\begin{description}
\item[{Returns:}] \leavevmode
list: list of bytes containing channel info.

\end{description}

\end{fulllineitems}

\index{exchange\_values() (reimaginedQuantum.Channel method)}

\begin{fulllineitems}
\phantomsection\label{\detokenize{code:reimaginedQuantum.Channel.exchange_values}}\pysiglinewithargsret{\sphinxbfcode{exchange\_values}}{\emph{read=True}}{}
Exchanges values from computer to utility.

Returns: can return None, or a list containing a \sphinxtitleref{hex\_list}

\end{fulllineitems}

\index{read\_value() (reimaginedQuantum.Channel method)}

\begin{fulllineitems}
\phantomsection\label{\detokenize{code:reimaginedQuantum.Channel.read_value}}\pysiglinewithargsret{\sphinxbfcode{read\_value}}{\emph{hex\_list}}{}
Reads a \sphinxtitleref{hex\_list} and updates class attributes values.

\end{fulllineitems}

\index{set\_value() (reimaginedQuantum.Channel method)}

\begin{fulllineitems}
\phantomsection\label{\detokenize{code:reimaginedQuantum.Channel.set_value}}\pysiglinewithargsret{\sphinxbfcode{set\_value}}{\emph{value}}{}
Writes and incoming int value to class attributes.

\end{fulllineitems}

\index{split\_value() (reimaginedQuantum.Channel method)}

\begin{fulllineitems}
\phantomsection\label{\detokenize{code:reimaginedQuantum.Channel.split_value}}\pysiglinewithargsret{\sphinxbfcode{split\_value}}{}{}
Updates the most/least significant byte.

\end{fulllineitems}

\index{verify\_values() (reimaginedQuantum.Channel method)}

\begin{fulllineitems}
\phantomsection\label{\detokenize{code:reimaginedQuantum.Channel.verify_values}}\pysiglinewithargsret{\sphinxbfcode{verify\_values}}{\emph{hex\_list}}{}
Verifies if current values have changed.
\begin{description}
\item[{Returns:}] \leavevmode
bool: The return value. True if values are the same, False otherwise.

\end{description}

\end{fulllineitems}


\end{fulllineitems}

\index{CommunicationPort (class in reimaginedQuantum)}

\begin{fulllineitems}
\phantomsection\label{\detokenize{code:reimaginedQuantum.CommunicationPort}}\pysiglinewithargsret{\sphinxstrong{class }\sphinxcode{reimaginedQuantum.}\sphinxbfcode{CommunicationPort}}{\emph{device}, \emph{baudrate=115200}, \emph{timeout=0.02}, \emph{bounce\_timeout=20}}{}
Constants
\index{BAUDRATE (reimaginedQuantum.CommunicationPort attribute)}

\begin{fulllineitems}
\phantomsection\label{\detokenize{code:reimaginedQuantum.CommunicationPort.BAUDRATE}}\pysigline{\sphinxbfcode{BAUDRATE}\sphinxstrong{ = 115200}}
Default baudrate for the serial port communication

\end{fulllineitems}

\index{BOUNCE\_TIMEOUT (reimaginedQuantum.CommunicationPort attribute)}

\begin{fulllineitems}
\phantomsection\label{\detokenize{code:reimaginedQuantum.CommunicationPort.BOUNCE_TIMEOUT}}\pysigline{\sphinxbfcode{BOUNCE\_TIMEOUT}\sphinxstrong{ = 20}}
Number of times a specific transmition is tried

\end{fulllineitems}

\index{BYTE\_SIZE (reimaginedQuantum.CommunicationPort attribute)}

\begin{fulllineitems}
\phantomsection\label{\detokenize{code:reimaginedQuantum.CommunicationPort.BYTE_SIZE}}\pysigline{\sphinxbfcode{BYTE\_SIZE}\sphinxstrong{ = 8}}
One byte = 8 bits

\end{fulllineitems}

\index{PARITY (reimaginedQuantum.CommunicationPort attribute)}

\begin{fulllineitems}
\phantomsection\label{\detokenize{code:reimaginedQuantum.CommunicationPort.PARITY}}\pysigline{\sphinxbfcode{PARITY}\sphinxstrong{ = `N'}}
Message will not have any parity

\end{fulllineitems}

\index{STOP\_BITS (reimaginedQuantum.CommunicationPort attribute)}

\begin{fulllineitems}
\phantomsection\label{\detokenize{code:reimaginedQuantum.CommunicationPort.STOP_BITS}}\pysigline{\sphinxbfcode{STOP\_BITS}\sphinxstrong{ = 1}}
Message contains only one stop bit

\end{fulllineitems}

\index{TIMEOUT (reimaginedQuantum.CommunicationPort attribute)}

\begin{fulllineitems}
\phantomsection\label{\detokenize{code:reimaginedQuantum.CommunicationPort.TIMEOUT}}\pysigline{\sphinxbfcode{TIMEOUT}\sphinxstrong{ = 0.02}}
Maximum time without answer from the serial port

\end{fulllineitems}

\index{begin\_serial() (reimaginedQuantum.CommunicationPort method)}

\begin{fulllineitems}
\phantomsection\label{\detokenize{code:reimaginedQuantum.CommunicationPort.begin_serial}}\pysiglinewithargsret{\sphinxbfcode{begin\_serial}}{}{}
Initializes pyserial instance.
\begin{description}
\item[{Returns:}] \leavevmode
pyserial.serial object

\item[{Raises:}] \leavevmode
PermissionError: user is not allowed to use port.
SerialException: if it could not open port

\end{description}

\end{fulllineitems}

\index{checksum() (reimaginedQuantum.CommunicationPort method)}

\begin{fulllineitems}
\phantomsection\label{\detokenize{code:reimaginedQuantum.CommunicationPort.checksum}}\pysiglinewithargsret{\sphinxbfcode{checksum}}{\emph{hex\_list}}{}
Implements a simple checksum to verify message integrity.
\begin{description}
\item[{Returns:}] \leavevmode
bool: The return value. True for success, False otherwise.

\end{description}

\end{fulllineitems}

\index{message() (reimaginedQuantum.CommunicationPort method)}

\begin{fulllineitems}
\phantomsection\label{\detokenize{code:reimaginedQuantum.CommunicationPort.message}}\pysiglinewithargsret{\sphinxbfcode{message}}{\emph{content}, \emph{wait\_for\_answer=False}}{}
Sends a message, and waits for answer.
\begin{description}
\item[{Returns:}] \leavevmode\begin{description}
\item[{list: each postion on list is made up with a tuple containing}] \leavevmode
channel and value in hexadecimal base.

\end{description}

\item[{Raises:}] \leavevmode
Exception: any type ocurred with during \sphinxtitleref{bounce\_timeout}.

\end{description}

\end{fulllineitems}

\index{read() (reimaginedQuantum.CommunicationPort method)}

\begin{fulllineitems}
\phantomsection\label{\detokenize{code:reimaginedQuantum.CommunicationPort.read}}\pysiglinewithargsret{\sphinxbfcode{read}}{}{}
Reads a message through the serial port.
\begin{description}
\item[{Returns:}] \leavevmode
list: hexadecimal values decoded as strings.

\item[{Raises:}] \leavevmode
Exception: Noisy answer, or timeout.

\end{description}

\end{fulllineitems}

\index{receive() (reimaginedQuantum.CommunicationPort method)}

\begin{fulllineitems}
\phantomsection\label{\detokenize{code:reimaginedQuantum.CommunicationPort.receive}}\pysiglinewithargsret{\sphinxbfcode{receive}}{}{}
Organices information according to project requirements.
\begin{description}
\item[{Returns:}] \leavevmode\begin{description}
\item[{list: each position on list is made up with a tuple containing}] \leavevmode
channel and value in hexadecimal base.

\end{description}

\item[{Raises:}] \leavevmode
Exception: if wrong checksum.

\end{description}

\end{fulllineitems}

\index{send() (reimaginedQuantum.CommunicationPort method)}

\begin{fulllineitems}
\phantomsection\label{\detokenize{code:reimaginedQuantum.CommunicationPort.send}}\pysiglinewithargsret{\sphinxbfcode{send}}{\emph{content}}{}
Sends a message through the serial port.
\begin{description}
\item[{Raises:}] \leavevmode
PySerialExceptions

\end{description}

\end{fulllineitems}


\end{fulllineitems}

\index{DataChannel (class in reimaginedQuantum)}

\begin{fulllineitems}
\phantomsection\label{\detokenize{code:reimaginedQuantum.DataChannel}}\pysiglinewithargsret{\sphinxstrong{class }\sphinxcode{reimaginedQuantum.}\sphinxbfcode{DataChannel}}{\emph{prefix}, \emph{port}}{}
Constants

\end{fulllineitems}

\index{Detector (class in reimaginedQuantum)}

\begin{fulllineitems}
\phantomsection\label{\detokenize{code:reimaginedQuantum.Detector}}\pysiglinewithargsret{\sphinxstrong{class }\sphinxcode{reimaginedQuantum.}\sphinxbfcode{Detector}}{\emph{identifier}, \emph{port}, \emph{data\_interval=100}, \emph{timer\_check\_interval=1000}}{}
Constants
\index{BASE\_DELAY (reimaginedQuantum.Detector attribute)}

\begin{fulllineitems}
\phantomsection\label{\detokenize{code:reimaginedQuantum.Detector.BASE_DELAY}}\pysigline{\sphinxbfcode{BASE\_DELAY}\sphinxstrong{ = 1e-09}}
Default channnel delay time (seconds)

\end{fulllineitems}

\index{BASE\_SLEEP (reimaginedQuantum.Detector attribute)}

\begin{fulllineitems}
\phantomsection\label{\detokenize{code:reimaginedQuantum.Detector.BASE_SLEEP}}\pysigline{\sphinxbfcode{BASE\_SLEEP}\sphinxstrong{ = 1e-09}}
Default channel sleep time (seconds)

\end{fulllineitems}


\end{fulllineitems}

\index{Experiment (class in reimaginedQuantum)}

\begin{fulllineitems}
\phantomsection\label{\detokenize{code:reimaginedQuantum.Experiment}}\pysiglinewithargsret{\sphinxstrong{class }\sphinxcode{reimaginedQuantum.}\sphinxbfcode{Experiment}}{\emph{port}, \emph{number\_detectors=2}}{}
Constants
\index{BASE\_COINWIN (reimaginedQuantum.Experiment attribute)}

\begin{fulllineitems}
\phantomsection\label{\detokenize{code:reimaginedQuantum.Experiment.BASE_COINWIN}}\pysigline{\sphinxbfcode{BASE\_COINWIN}\sphinxstrong{ = 1e-09}}
Default coincidence window (seconds)

\end{fulllineitems}

\index{BASE\_SAMPLING (reimaginedQuantum.Experiment attribute)}

\begin{fulllineitems}
\phantomsection\label{\detokenize{code:reimaginedQuantum.Experiment.BASE_SAMPLING}}\pysigline{\sphinxbfcode{BASE\_SAMPLING}\sphinxstrong{ = 0.001}}
Default sampling time (seconds)

\end{fulllineitems}


\end{fulllineitems}

\index{TimerChannel (class in reimaginedQuantum)}

\begin{fulllineitems}
\phantomsection\label{\detokenize{code:reimaginedQuantum.TimerChannel}}\pysiglinewithargsret{\sphinxstrong{class }\sphinxcode{reimaginedQuantum.}\sphinxbfcode{TimerChannel}}{\emph{prefix}, \emph{port}, \emph{base}}{}
Constants

\end{fulllineitems}



\section{Graphical User Interface}
\label{\detokenize{code:graphical-user-interface}}\label{\detokenize{code:module-mainGUI}}\index{mainGUI (module)}
Created on Tue Apr 11 11:31:32 2017

@author: juan
\index{AutoSizeLabel (class in mainGUI)}

\begin{fulllineitems}
\phantomsection\label{\detokenize{code:mainGUI.AutoSizeLabel}}\pysiglinewithargsret{\sphinxstrong{class }\sphinxcode{mainGUI.}\sphinxbfcode{AutoSizeLabel}}{\emph{text}, \emph{value}}{}
from reclosedev at \url{http://stackoverflow.com/questions/8796380/automatically-resizing-label-text-in-qt-strange-behaviour}
and Jean-Sébastien \url{http://stackoverflow.com/questions/29852498/syncing-label-fontsize-with-layout-in-pyqt}

\end{fulllineitems}

\index{Main (class in mainGUI)}

\begin{fulllineitems}
\phantomsection\label{\detokenize{code:mainGUI.Main}}\pysigline{\sphinxstrong{class }\sphinxcode{mainGUI.}\sphinxbfcode{Main}}~\begin{quote}

Defines the mainwindow.
\end{quote}

Constants
\index{channelsCaller() (mainGUI.Main method)}

\begin{fulllineitems}
\phantomsection\label{\detokenize{code:mainGUI.Main.channelsCaller}}\pysiglinewithargsret{\sphinxbfcode{channelsCaller}}{}{}
creates a property window to define number of channels

\end{fulllineitems}

\index{choose\_file() (mainGUI.Main method)}

\begin{fulllineitems}
\phantomsection\label{\detokenize{code:mainGUI.Main.choose_file}}\pysiglinewithargsret{\sphinxbfcode{choose\_file}}{}{}
user interaction with saving file

\end{fulllineitems}

\index{eventFilter() (mainGUI.Main method)}

\begin{fulllineitems}
\phantomsection\label{\detokenize{code:mainGUI.Main.eventFilter}}\pysiglinewithargsret{\sphinxbfcode{eventFilter}}{\emph{source}, \emph{event}}{}
Creates event to handle serial combobox opening.

\end{fulllineitems}

\index{file\_changed (mainGUI.Main attribute)}

\begin{fulllineitems}
\phantomsection\label{\detokenize{code:mainGUI.Main.file_changed}}\pysigline{\sphinxbfcode{file\_changed}\sphinxstrong{ = None}}
set

\end{fulllineitems}

\index{format (mainGUI.Main attribute)}

\begin{fulllineitems}
\phantomsection\label{\detokenize{code:mainGUI.Main.format}}\pysigline{\sphinxbfcode{format}\sphinxstrong{ = None}}
fig

\end{fulllineitems}

\index{select\_serial() (mainGUI.Main method)}

\begin{fulllineitems}
\phantomsection\label{\detokenize{code:mainGUI.Main.select_serial}}\pysiglinewithargsret{\sphinxbfcode{select\_serial}}{\emph{index}}{}
Selects port at index position of combobox.

\end{fulllineitems}

\index{serial\_refresh() (mainGUI.Main method)}

\begin{fulllineitems}
\phantomsection\label{\detokenize{code:mainGUI.Main.serial_refresh}}\pysiglinewithargsret{\sphinxbfcode{serial\_refresh}}{}{}
Loads serial port described at user combobox.

\end{fulllineitems}

\index{widget\_activate() (mainGUI.Main method)}

\begin{fulllineitems}
\phantomsection\label{\detokenize{code:mainGUI.Main.widget_activate}}\pysiglinewithargsret{\sphinxbfcode{widget\_activate}}{\emph{status}}{}
most of the tools will be disabled if there is no UART detected

\end{fulllineitems}


\end{fulllineitems}

\index{RingBuffer (class in mainGUI)}

\begin{fulllineitems}
\phantomsection\label{\detokenize{code:mainGUI.RingBuffer}}\pysiglinewithargsret{\sphinxstrong{class }\sphinxcode{mainGUI.}\sphinxbfcode{RingBuffer}}{\emph{rows}, \emph{columns}, \emph{output\_file}, \emph{fmt}, \emph{delimiter='t'}}{}
Based on \url{https://scimusing.wordpress.com/2013/10/25/ring-buffers-in-pythonnumpy/}
\index{extend() (mainGUI.RingBuffer method)}

\begin{fulllineitems}
\phantomsection\label{\detokenize{code:mainGUI.RingBuffer.extend}}\pysiglinewithargsret{\sphinxbfcode{extend}}{\emph{x}}{}
adds array x to ring buffer

\end{fulllineitems}

\index{get() (mainGUI.RingBuffer method)}

\begin{fulllineitems}
\phantomsection\label{\detokenize{code:mainGUI.RingBuffer.get}}\pysiglinewithargsret{\sphinxbfcode{get}}{}{}
Returns the first-in-first-out data in the ring buffer

\end{fulllineitems}

\index{save() (mainGUI.RingBuffer method)}

\begin{fulllineitems}
\phantomsection\label{\detokenize{code:mainGUI.RingBuffer.save}}\pysiglinewithargsret{\sphinxbfcode{save}}{}{}
Saves the buffer

\end{fulllineitems}


\end{fulllineitems}

\index{heavy\_import() (in module mainGUI)}

\begin{fulllineitems}
\phantomsection\label{\detokenize{code:mainGUI.heavy_import}}\pysiglinewithargsret{\sphinxcode{mainGUI.}\sphinxbfcode{heavy\_import}}{}{}
Imports matplotlib and NumPy.

Useful to be combined with threading processes.

\end{fulllineitems}

\index{propertiesWindow (class in mainGUI)}

\begin{fulllineitems}
\phantomsection\label{\detokenize{code:mainGUI.propertiesWindow}}\pysiglinewithargsret{\sphinxstrong{class }\sphinxcode{mainGUI.}\sphinxbfcode{propertiesWindow}}{\emph{parent=None}}{}
Defines the channel configuration dialog.
\index{DEFAULT\_CHANNELS (mainGUI.propertiesWindow attribute)}

\begin{fulllineitems}
\phantomsection\label{\detokenize{code:mainGUI.propertiesWindow.DEFAULT_CHANNELS}}\pysigline{\sphinxbfcode{DEFAULT\_CHANNELS}\sphinxstrong{ = 2}}
Default number of channels

\end{fulllineitems}

\index{creator() (mainGUI.propertiesWindow method)}

\begin{fulllineitems}
\phantomsection\label{\detokenize{code:mainGUI.propertiesWindow.creator}}\pysiglinewithargsret{\sphinxbfcode{creator}}{\emph{n}}{}
creates the spinboxes and labels required by the user

\end{fulllineitems}

\index{delete() (mainGUI.propertiesWindow method)}

\begin{fulllineitems}
\phantomsection\label{\detokenize{code:mainGUI.propertiesWindow.delete}}\pysiglinewithargsret{\sphinxbfcode{delete}}{\emph{n}, \emph{N}}{}
delets unneccesary rows of labels and spinboxes

\end{fulllineitems}

\index{reset() (mainGUI.propertiesWindow method)}

\begin{fulllineitems}
\phantomsection\label{\detokenize{code:mainGUI.propertiesWindow.reset}}\pysiglinewithargsret{\sphinxbfcode{reset}}{}{}
sets everything to default

\end{fulllineitems}

\index{update() (mainGUI.propertiesWindow method)}

\begin{fulllineitems}
\phantomsection\label{\detokenize{code:mainGUI.propertiesWindow.update}}\pysiglinewithargsret{\sphinxbfcode{update}}{}{}
sends message with the updated information

\end{fulllineitems}


\end{fulllineitems}

\index{savetxt() (in module mainGUI)}

\begin{fulllineitems}
\phantomsection\label{\detokenize{code:mainGUI.savetxt}}\pysiglinewithargsret{\sphinxcode{mainGUI.}\sphinxbfcode{savetxt}}{\emph{file}, \emph{matrix}, \emph{delimiter='}, \emph{`}, \emph{fmt='\%.3f'}, \emph{typ=\textless{}class `float'\textgreater{}}}{}
Saves data to a text file.

Used to save matrix contents to plain text files. 
Depening whether or not matrix contains strings or floats
uses np.savetxt function.

\end{fulllineitems}



\chapter{Indices and tables}
\label{\detokenize{index:indices-and-tables}}\begin{itemize}
\item {} 
\DUrole{xref,std,std-ref}{genindex}

\item {} 
\DUrole{xref,std,std-ref}{modindex}

\end{itemize}


\renewcommand{\indexname}{Python Module Index}
\begin{sphinxtheindex}
\def\bigletter#1{{\Large\sffamily#1}\nopagebreak\vspace{1mm}}
\bigletter{m}
\item {\sphinxstyleindexentry{mainGUI}}\sphinxstyleindexpageref{code:\detokenize{module-mainGUI}}
\indexspace
\bigletter{r}
\item {\sphinxstyleindexentry{reimaginedQuantum}}\sphinxstyleindexpageref{code:\detokenize{module-reimaginedQuantum}}
\end{sphinxtheindex}

\renewcommand{\indexname}{Index}
\printindex
\end{document}